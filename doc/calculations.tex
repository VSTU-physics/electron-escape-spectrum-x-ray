\section{Модуль calculations}

В этом модуле собраны функции, реализующие численные методы решения дифференциальных уравнений в частных производных и интерполирования.

\begin{verbatim}
void spe(double *f,
         double *f_prev,
         double *z,
         int N,
         double dt,
         double pt,
         double gt,
         double a_1t,
         double b_1t,
         double c_1t,
         double a_2t,
         double b_2t,
         double c_2t)
\end{verbatim}

spe решает уравнение:
\[
        \frac{\partial f}{\partial t} =
        p(z, t) \frac{\partial^2 f}{\partial z^2} + g(z, t),
\]
при граничных условиях:
\[
        a_1(t) f + b_1(t) \frac{\partial f}{\partial z} \Bigg|_{z = z_
1} = c_1(t),
\]
\[
        a_2(t) f + b_2(t) \frac{\partial f}{\partial z} \Bigg|_{z = z_
2} = c_2(t),
\]
и произвольных начальных условиях неявным методом на одном шаге. Ошибка $O(h)$. A, B, C -- диагонали матрицы нижняя, собственно диагональ, верхняя. D -- правая часть уравнения.

\begin{verbatim}
void cubic_spline(double *x, double *y, int N, double *a, double *b, double *c, double *d)
\end{verbatim}

\begin{verbatim}
void eval_cubic_spline(double *xs, double *ys, int M, double *x, double *y, int N)
\end{verbatim}

\verb+eval_cubic_spline+ заносит значения в точках xs[j] в массив ys[j], j = 0,
1, 2, ..., M-1 функции заданной кубиче-
скими сплайнами для таблицы (x[i], y[i]), i = 0, 1, 2, ... , N-1. 

\verb+int_cubic_spline+ ищет интегралы, \verb+cubic_spline+ ищет коэффициенты сплайна:
\[
    S_i(x) = a_i + b_i (x - x_i) + \frac{1}{2} c_i (x - x_i)^2 + \frac{1}{6} d_i (x - x_i)^3
    \quad x\in[x_i, x_{i+1}].
\]


\begin{verbatim}
double int_cubic_spline(double la, double lb, double *x, double *y, int N)
\end{verbatim}

\begin{verbatim}
double random(double l)
\end{verbatim}

\begin{verbatim}
double linterp(double x, double x1, double y1, double x2, double y2)
\end{verbatim}

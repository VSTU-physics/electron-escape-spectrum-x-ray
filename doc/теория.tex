\documentclass[a4paper,12pt]{article} %размер бумаги устанавливаем А4, шрифт 12пунктов
\usepackage[T2A]{fontenc}
\usepackage[utf8]{inputenc}%включаем свою кодировку: koi8-r или utf8 в UNIX, cp1251 в

\usepackage[english,russian]{babel}%используем русский и английский языки с переносами
\usepackage{amssymb,amsfonts,amsmath,mathtext,cite,enumerate,float,bm} %подключаем
\usepackage[colorlinks=True,urlcolor=blue]{hyperref}
%\usepackage{caption2} %Чтобы поменять двоеточие в названии таблицы на тире
%\renewcommand{\rmdefault}{ftm}

\usepackage{geometry}
\geometry{left=1cm}
\geometry{right=1cm}
\geometry{top=1cm}
\geometry{bottom=2cm}

\renewcommand{\baselinestretch}{1.5}  % 1 интервал

\renewcommand{\thetable}{\arabic{section}.\arabic{table}}

%\renewcommand{\captionlabeldelim}{ \textendash}

\tolerance=10000

\newcommand{\D}{\mathrm{d}}
% Производные
\newcommand{\dsl}[2]{{\partial #1}/{\partial #2}}
\newcommand{\df}[1]{\cfrac{\partial}{\partial #1}}
\newcommand{\dff}[2]{\frac{\partial #1}{\partial #2}}
\newcommand{\dfs}[2]{\frac{\partial^2 #1}{\partial #2^2}}
\newcommand{\Df}[1]{\frac{d}{d #1}}
\newcommand{\Dff}[2]{\frac{d #1}{d #2}}
\newcommand{\Dfs}[2]{\frac{d^2 #1}{d #2^2}}
\newcommand{\cDf}[1]{\cfrac{d}{d #1}}
\newcommand{\cDff}[2]{\cfrac{d #1}{d #2}}
\newcommand{\cDfs}[2]{\cfrac{d^2 #1}{d #2^2}}
\newcommand{\dfn}[3]{\frac{\partial^#1 #2}{\partial #3^#1}}
% Векторы
\renewcommand{\vec}[1]{\bm{#1}}
\newcommand{\ort}[1]{\bm{\mathrm{e}}_#1}
% Векторный анализ
\renewcommand{\div}{\mathrm{div}\,}
\newcommand{\rot}{\mathrm{rot}\,}
\newcommand{\grad}{\mathrm{grad}\,}
\newcommand{\laplas}[4]{\dfs{#1}{#2}+\dfs{#1}{#3}+\dfs{#1}{#4}}
\newcommand{\laplasxyz}[1]{\dfs{#1}{x}+\dfs{#1}{y}+\dfs{#1}{z}}
\newcommand{\rotc}[4]{\dff{#1}{#2} - \dff{#3}{#4}}
\newcommand{\rotcx}[3]{\dff{#1\vphantom{E}_#3}{#2} - \dff{#1\vphantom{E}_#2}{#3}}
% Функции
\renewcommand{\cosh}{\mathrm{ch}\,}
\renewcommand{\sinh}{\mathrm{sh}\,}
\renewcommand{\tanh}{\mathrm{th}\,}
\renewcommand{\Im}{\mathrm{Im}\,}
\renewcommand{\Re}{\mathrm{Re}\,}
\renewcommand{\det}[4]{#1 #4 - #2 #3}
\renewcommand{\matrix}[4]{\begin{pmatrix}#1 & #2 \\ #3 & #4\end{pmatrix}}
\newcommand{\matrixw}[5]{\begin{#5matrix}#1 & #2 \\ #3 & #4\end{#5matrix}} % #5 = b p s v V
\newcommand{\col}[2]{\begin{pmatrix}#1 & #2\end{pmatrix}}
\newcommand{\colw}[3]{\begin{#3matrix}#1 & #2 \end{#3matrix}} % #3 = b p s v V
\newcommand{\row}[2]{\begin{pmatrix}#1 \\ #2\end{pmatrix}}
\newcommand{\roww}[3]{\begin{#3matrix}#1 \\ #2 \end{#3matrix}} % #3 = b p s v V
\newcommand{\matrixrot}[2]{\begin{#2matrix} \ort{x} & \ort{y} & \ort{z} \\ \df{x} & \df{y} & \df{z} \\ #1_x & #1_y & #1_z \end{#2matrix}}
\newcommand{\matrixrotr}[4]{\begin{#4matrix} \ort{x} & \ort{y} & \ort{z} \\ \df{x} & \df{y} & \df{z} \\ #1 & #2 & #3 \end{#4matrix}}
\newcommand{\eps}{\varepsilon}
\renewcommand{\phi}{\varphi}

\newcommand{\shtr}{\mathop{\!\vphantom{E}'}}
\newcommand{\ind}[1]{\mathop{\!\vphantom{E}_{#1}}}
%\renewcommand{\theequation}{\arabic{section}.\arabic{equation}}

\usepackage[pdftex]{graphicx, color}
\usepackage{enumitem}

\setlist[itemize]{noitemsep, topsep=0pt, parsep=0pt}
\begin{document}

\section{Что решаем?}

Требуется решить уравнение:
\begin{equation}\label{eq: main}
\begin{gathered}
    \vec{\Omega}\cdot\nabla F(\vec{r}, \vec{\Omega}, E) =
    \df{E}[\bar{\eps}(E)F(\vec{r}, \vec{\Omega}, E)] + \\ +
    \int\limits_{4\pi} \lambda^{-1}_{el}(E, \vec{\Omega}', \vec{\Omega}) \big[ F(\vec{r}, \vec{\Omega}', E) - F(\vec{r}, \vec{\Omega}, E) \big] d\Omega' +
    Q(\vec{r}, \vec{\Omega}, E),
\end{gathered}
\end{equation}
где
\begin{itemize}
    \item[--] $\vec{\Omega}$ -- единичный вектор;
    \item[--] $F(\vec{r}, \vec{\Omega}, E)$ --  вероятность найти частицу в окрестности $dV$ вектора $\vec{r}$, движущуюся в телесном угле $d\Omega$ в направлении $\vec{\Omega}$, с энергией в диапазоне $E, E+dE$;
    \item[--] $\bar{\eps}(E)$ -- средние потери на единице пути:
    \begin{equation}\label{eq: sharp}
        \bar{\eps}(E) = \int\limits_{0}^{E/2} \lambda_{in}^{-1}(E, \eps) \eps d\eps;
    \end{equation}
    \item[--] $\lambda_{in}^{-1}(E, \eps)$ -- вероятность потерять на единице длины энергию $\eps$ (вероятность неупругого столкновения);
    \item[--] $\lambda^{-1}_{el}(E, \vec{\Omega}', \vec{\Omega})$ -- вероятность изменения направления движения с $\vec{\Omega}'$ на $\vec{\Omega}$ на единице пути (вероятность упругого столкновения);
    \item[--] $Q(\vec{r}, \vec{\Omega}, E)$ -- функция источника, для Оже-процессов имеет вид:
    \begin{equation}
        Q(\vec{r}, \vec{\Omega}, E) = \frac{1}{4\pi} \sum_{i} P_i\delta(E - E_i);
    \end{equation}
    \item[--] $P_i$ -- вероятность рождения Оже-электрона с энергией $E_i$.
\end{itemize}

$P_1$ приближение:
\begin{gather}\label{eq: P1}
    F(\vec{r}, \vec{\Omega}, E) = \frac{1}{4\pi} [ F_0(\vec{r}, E) + 3 \vec{\Omega}\cdot\vec{F_1}(\vec{r}, E) ].
\end{gather}

Здесь:
\begin{align}
    & F_0(\vec{r}, E) = \int\limits_{4\pi} F(\vec{r}, \vec{\Omega}, E) d\Omega, \\
    & \vec{F}_1(\vec{r}, E) = \int\limits_{4\pi} \vec{\Omega} F(\vec{r}, \vec{\Omega}, E) d\Omega.
\end{align}

Подставим \eqref{eq: P1} в \eqref{eq: main}, получим:
\begin{equation}\label{eq: P1is}
\begin{gathered}
    \vec{\Omega}\cdot\nabla [ F_0(\vec{r}, E) + 3 \vec{\Omega}\cdot\vec{F_1}(\vec{r}, E) ] =
    \df{E}[\bar{\eps}(E)[ F_0(\vec{r}, E) + 3 \vec{\Omega}\cdot\vec{F_1}(\vec{r}, E) ]] + \\ +
    \int\limits_{4\pi} \lambda^{-1}_{el}(E, \vec{\Omega}', \vec{\Omega}) 3 \big[ \vec{\Omega}' - \vec{\Omega}\big] \cdot \vec{F}_1 (\vec{r}, E) d\Omega' +
    \sum_{i} P_i\delta(E - E_i).
\end{gathered}
\end{equation}

Для упругого столкновения $\lambda^{-1}_{el}(E, \vec{\Omega}', \vec{\Omega})$ зависит только от угла между $\vec{\Omega}'$, $\vec{\Omega}$ -- $\theta$. Рассмотрим детально интеграл:
\begin{equation}
\begin{gathered}
    \int\limits_{4\pi} \lambda^{-1}_{el}(E, \vec{\Omega}', \vec{\Omega}) 3 \big[ \vec{\Omega}' - \vec{\Omega}\big] d\Omega' =
    \int\limits_{4\pi} \lambda^{-1}_{el}(E, \theta) 3 \big[ \vec{\Omega}'_{\perp} + \vec{\Omega} \cos \theta - \vec{\Omega}\big] \sin d\theta d\phi = \\
    = [\text{ интеграл от } \vec{\Omega}'_{\perp} \text{ по } \phi ~ = 0] =
    6\pi\int\limits_{0}^\pi \lambda^{-1}_{el}(E, \theta) \big[ \cos \theta - 1\big] \sin \theta d\theta \vec{\Omega} = \\ =
    - 3 \lambda^{-1}_{tr} \vec{\Omega},
\end{gathered}
\end{equation}
где обозначено:
\begin{equation}
    \lambda^{-1}_{tr}(E) = 2\pi\int\limits_{0}^\pi \lambda^{-1}_{el}(E, \theta) \big[ 1 -\cos \theta \big] \sin \theta d\theta.
\end{equation}
$\lambda_{tr}$ носит название транспортной длины.

Далее рассмотрим выражение
\begin{equation}
\begin{gathered}
    \vec{\Omega}\cdot\nabla [\vec{\Omega}\cdot\vec{F_1}] =
    \vec{\Omega}\cdot[
    \vec{\Omega} \times \rot \vec{F_1} +
    (\vec{\Omega}\cdot\nabla) \vec{F_1} + 0
    ] = \vec{\Omega}(\vec{\Omega}\cdot\nabla)\cdot \vec{F_1} = \\
    = (\vec{\Omega}\times(\vec{\Omega}\times\nabla) + \nabla (\vec{\Omega}\cdot\vec{\Omega}))\cdot\vec{F_1} =
    (\vec{\Omega}\times(\vec{\Omega}\times\nabla) + \nabla)\cdot\vec{F_1}
\end{gathered}
\end{equation}

\colorbox{yellow}{По какой-то причине мы выкидываем $\vec{\Omega}\times(\vec{\Omega}\times\nabla)$.}

Направление $\vec{\Omega}$ произвольно, поэтому из линейной независимости в \eqref{eq: P1is} следует:
\begin{align}
    & \vec{F}_1 = -\frac{\lambda_{tr}}{3}\nabla F_0 + \lambda_{tr} \dff{\bar{\eps} \vec{F_1}}{E}, \label{eq: P1-1}\\
    & \nabla\cdot\vec{F}_1 = \dff{\bar{\eps}F_0}{E} + \sum_{i} P_i\delta(E - E_i). \label{eq: P1-2}
\end{align}

В диффузионном приближении пренебрегают последним слагаемым в \eqref{eq: P1-1}, тогда из [\ref{eq: P1-1}-\ref{eq: P1-2}] следует:
\begin{equation}
    \dff{(\bar{\eps}F_0)}{E} + \sum_{i} P_i\delta(E - E_i) = -\frac{\lambda_{tr}}{3}\Delta F_0
\end{equation}

\clearpage

\section{Граничные условия}

Физический смысл $\vec{F}_1$ -- поток. В результате поток через границу в точке с нормалью $\vec{n}_\Gamma$ и коэффициентом пропускания $k_D$ будет равен:
\begin{equation}
    \vec{n}_\Gamma \cdot \vec{F}_1 =
    \int\limits_{\vec{n}_\Gamma \cdot \vec{\Omega}>0} k_D(\vec{\Omega}) \vec{n}_\Gamma \cdot \vec{\Omega} F(\vec{r}, \vec{\Omega}, E) d\Omega = \frac{1}{4\pi} \int\limits_{\vec{n}_\Gamma \cdot \vec{\Omega}>0} k_D(\vec{\Omega}, \vec{n}_\Gamma) \vec{n}_\Gamma \cdot \vec{\Omega} [F_0(\vec{r}, E) + 3\vec{\Omega}\cdot\vec{F}_1(\vec{r}, E)]d\Omega
\end{equation}

$k_D$ зависит только от угла между векторами $\vec{\Omega}$, $\vec{n}_\Gamma$. Пусть этот угол -- $\theta$, тогда
\begin{equation}
    k_D(\theta, E) =
    \begin{cases}
        0,  & \arcsin\sqrt{1 - U_0/E} \leqslant \theta \leqslant \pi/2; \\
        1 - \Bigg(\cfrac{1 - \sqrt{1 - U_0/(E  \cos^2 \theta)}}{1 + \sqrt{1 - U_0/(E  \cos^2 \theta)}}\Bigg)^2, & 0 \leqslant \theta < \arcsin\sqrt{1 - U_0/E}
    \end{cases}
\end{equation}

\begin{equation}
\begin{gathered}
    \frac{1}{4\pi} \int\limits_{\vec{n}_\Gamma \cdot \vec{\Omega}>0} k_D(\vec{\Omega}, \vec{n}_\Gamma) \vec{n}_\Gamma \cdot \vec{\Omega} [F_0(\vec{r}, E) + 3\vec{\Omega}\cdot\vec{F}_1(\vec{r}, E)]d\Omega = \\
    =
    \frac{1}{4\pi}
    \int\limits_{0}^{2\pi}
    \int\limits_{0}^{\pi/2}
     k_D(\theta, E) \cos \theta [F_0(\vec{r}, E) + 3(\cos \theta \vec{n}_\Gamma + \vec{\Omega}_\perp)\cdot \vec{F}_1(\vec{r}, E)]
     \sin \theta d\theta d\phi = \\
     =
     [\text{интеграл по }\phi~=~2\pi] =
     \frac{1}{2}
     \int\limits_{0}^{\pi/2}
     k_D(\theta, E) \cos \theta [F_0(\vec{r}, E) + 3 \cos \theta \vec{n}_\Gamma \cdot \vec{F}_1(\vec{r}, E)]
     \sin \theta d\theta
\end{gathered}
\end{equation}

Найдём интегралы:
\begin{equation}
\begin{gathered}
    I_1 =
    \int\limits_{0}^{\theta_{max}}
    \left(1 - \Bigg(\cfrac{1 - \sqrt{1 - U_0/(E  \cos^2 \theta)}}{1 + \sqrt{1 - U_0/(E  \cos^2 \theta)}}\Bigg)^2\right) \cos \theta \sin \theta d\theta = \\
    = [U_0/E = \eps = \cos \theta_{max}] =
    \int\limits_{0}^{\theta_{max}}
    \left(1 - \Bigg(\cfrac{(1 - \sqrt{1 - \eps/\cos^2 \theta})^2}{\eps/ \cos^2 \theta}\Bigg)^2\right) \cos \theta \sin \theta d\theta = \\
    = [x = \cos \theta] =
    - \int\limits_{1}^{\eps}
    \left(1 - \Bigg(\cfrac{(x - \sqrt{x^2 - \eps})^2}{\eps}\Bigg)^2\right) x dx = \\
    =
    \int\limits_{\eps}^{1}
    \left(1 - \Bigg(\cfrac{2 x^2 - \eps - 2 x \sqrt{x^2 - \eps} }{\eps}\Bigg)^2\right) x dx = \\
    =
    \int\limits_{\eps}^{1}
    \left(1 - \cfrac{(2 x^2 - \eps)^2 - 4 x (2 x^2 - \eps) \sqrt{x^2 - \eps} + 4x^2(x^2 - \eps) }{\eps^2}\right) x dx = \\
    =
    \int\limits_{\eps}^{1}
    -\left(\cfrac{8x^4 - 8 x^2 \eps - 4 x (2 x^2 - \eps) \sqrt{x^2 - \eps}}{\eps^2}\right) x dx
    = \\ =
    - \frac{8}{6} \frac{1 - \eps^6}{\eps^2}
    + \frac{8}{4} \frac{1 - \eps^4}{\eps}
    + \int\limits_{\eps}^{1}
    \left(\cfrac{2 (2 x^2 - \eps) \sqrt{(x^2)^2 - \eps x^2}}{\eps^2}\right) dx^2 = \\ =
    - \frac{8}{6} \frac{1 - \eps^6}{\eps^2}
    + \frac{8}{4} \frac{1 - \eps^4}{\eps}
    + 2 \int\limits_{\eps}^{1}
    \cfrac{\sqrt{(x^2)^2 - \eps x^2}}{\eps^2} d((x^2)^2 - \eps x^2) = \\ =
    - \frac{8}{6} \frac{1 - \eps^6}{\eps^2}
    + \frac{8}{4} \frac{1 - \eps^4}{\eps}
    + \cfrac{4 (1 - \eps)^{3/2}}{ 3 \eps^2}
    - \cfrac{4 (\eps^2- \eps)^{3/2}}{ 3 \eps}
\end{gathered}
\end{equation}

\begin{equation}
\begin{gathered}
    I_2 =
    \int\limits_{0}^{\theta_{max}}
    \left(1 - \Bigg(\cfrac{1 - \sqrt{1 - U_0/(E  \cos^2 \theta)}}{1 + \sqrt{1 - U_0/(E  \cos^2 \theta)}}\Bigg)^2\right) \cos^2 \theta \sin \theta d\theta = \\
    = [U_0/E = \eps = \cos \theta_{max}, x = \cos \theta] =
    -\int\limits_{\eps}^{1}\left(\cfrac{8x^4 - 8 x^2 \eps - 4 x (2 x^2 - \eps) \sqrt{x^2 - \eps}}{\eps^2}\right) x^2 dx
    = \\ =
    - \frac{8}{7} \frac{1 - \eps^7}{\eps^2}
    + \frac{8}{5} \frac{1 - \eps^5}{\eps}
    + 2 \int\limits_{\eps}^{1}
    \cfrac{(2 x^2 - \eps) \sqrt{x^2 - \eps}}{\eps^2} x^2 dx^2 = \\ =
    [t = \sqrt{x^2 - \eps}] =
    - \frac{8}{7} \frac{1 - \eps^7}{\eps^2}
    + \frac{8}{5} \frac{1 - \eps^5}{\eps}
    + 2 \int\limits_{..}^{..}
    \cfrac{(2 t^2 + \eps) t}{\eps^2} (t^2 + \eps) dt^2 = \\ =
    - \frac{8}{7} \frac{1 - \eps^7}{\eps^2}
    + \frac{8}{5} \frac{1 - \eps^5}{\eps}
    + 4 \int\limits_{..}^{..}
    \cfrac{(2 t^6 + 3 \eps t^4 + \eps^2 t^2)}{\eps^2}  dt= \\ =
    - \frac{8}{7} \frac{1 - \eps^7}{\eps^2}
    + \frac{8}{5} \frac{1 - \eps^5}{\eps}
    + \frac{8}{7} \frac{(1 - \eps)^{7/2} - (\eps^2 - \eps)^{7/2}}{\eps^2}
    + \frac{12}{5} \frac{(1 - \eps)^{5/2} - (\eps^2 - \eps)^{5/2}}{\eps}
    + \frac{8}{3} ((1 - \eps)^{3/2} - (\eps^2 - \eps)^{3/2})
\end{gathered}
\end{equation}

\colorbox{yellow}{Сравним с тем, что в программе}
\begin{gather*}
I_1 =
\frac{1}{2} (\eps + 1/\eps) - 1 +
\frac{1}{4} (1 - 2/\eps)\sqrt{1 - \eps} -
\frac{\eps}{8} \ln \frac{\eps}{2 - \eps + 2 \sqrt{1 - \eps}}
\end{gather*}
\begin{gather*}
I_2 =
\frac{2}{3} (1 -\eps^{3/2})
+ \frac{2}{3} (1 -\eps)^{3/2}
- \frac{2}{5 \eps} (1 -\eps^{5/2})
+ \frac{2}{5 \eps} (1 -\eps)^{5/2}
\end{gather*}

Тогда граничное условие:
\begin{equation}
    \vec{n}_\Gamma \cdot \vec{F}_1 \left(1 - \frac{3}{2} I_2 \right) = \frac{1}{2} I_1 F_0
\end{equation}

В диффузионном приближении:
\begin{equation}
    \frac{1}{2} \frac{I_1}{1 - 3 I_2/2} F_0 = -\frac{\lambda_{tr}}{3}
    (\vec{n}_\Gamma \cdot \nabla) F_0
\end{equation}

\colorbox{yellow}{Странно, знак не совпал.}

\clearpage
\section{Окончательный вариант}

Уравнение записывается относительно плотности:
\begin{equation}
    n = \bar{\eps} F_0
\end{equation}

Вместо энергии выступает:
\begin{equation}
    t = \frac{E_{max} - E}{E_{max}},
\end{equation}
где $E_{max}$ -- максимальная энергия Оже-электронов.

Уравнение:
\begin{equation}
    \dff{n}{t} = E_{max}\frac{\lambda_{tr}}{3\bar{\eps}}\dfn{2}{n}{z} + \sum_{i} P_i\delta(1 - t - E_i/E_{max}), \quad t\in[0, 1].
\end{equation}

Граничные:
\begin{gather}
    \frac{1}{2} \frac{I_1}{1 - 3 I_2/2} n + \frac{\lambda_{tr}}{3} \dff{n}{z} \Bigg|_{z = 0} = 0 \\
    n\Bigg|_{z = l} = 0
\end{gather}

Начальные: \colorbox{yellow}{в программе это выглядит как ''Hello, yellow house''}
\begin{equation}
    n = P_1 \eta (z)
\end{equation}

\section{Аналитический метод в программе}

В этом случае $\lambda_{tr}^{-1}$ считается аналитически:
\begin{equation}\label{eq: ltr}
    \lambda_{tr}^{-1} =
    2 \pi \frac{\rho N_A}{M} \int\limits_{0}^{\pi}
    \Dff{\sigma_{el}}{\Omega} (1 - \cos \theta) \sin \theta d \theta.
\end{equation}

Модификация (кажется Бете) формулы для дифференциального сечения рассеяния (Резерфорда):
\begin{equation}
    \Dff{\sigma_{el}}{\Omega} =
    \frac{(T+1)^2 r_e^2}{T^2(T+2)^2} \frac{Z(Z+1)}{(1 - \cos \theta + 2 \eta)^2},
\end{equation}
где
\begin{itemize}
    \item[--] $T$ -- кинетическая энергия электрона, отнесённая к энергии покоя:
    \begin{equation}
        T = \frac{E}{mc^2};
    \end{equation}
    \item[--] $r_e$ -- классический радиус электрона:
    \begin{equation}
        r_e = \frac{e^2}{mc^2} (\text{В СГС}) = \frac{1}{4 \pi \eps_0}\frac{e^2}{mc^2} (\text{В СИ});
    \end{equation}
    \item[--] $\eta$ -- поправка:
    \begin{equation}
        \eta = \frac{1}{4}
        \Big[
            \frac{\alpha Z^{1/3}}{0{,}885}
        \Big]^2
        \frac{1}{T(T+2)} \left(1{,}13 + 3{,}76 \alpha^2 Z^2 \frac{(T+1)^2}{T(T+2)} \right);
    \end{equation}
    \item[--] $\alpha$ -- постоянная тонкой структуры:
    \begin{equation}
        \alpha = \frac{e^2}{\hbar c} (\text{В СГС}) = \frac{1}{4 \pi \eps_0} \frac{e^2}{\hbar c} (\text{В СИ});
    \end{equation}
    \item[--] $\rho$, $N_A$, $M$ -- плотность вещества, число Авогадро, молярная масса.
\end{itemize}

После интегрирования:
\begin{equation}
    \begin{gathered}
        \lambda_{tr}^{-1} =
        2 \pi \frac{\rho N_AZ(Z+1)}{M}
        \frac{(T+1)^2 r_e^2}{T^2(T+2)^2}
        \int\limits_{0}^{\pi}
        \frac{1}{(1 - \cos \theta + 2 \eta)^2} (1 - \cos \theta) \sin \theta d \theta
        = \\ =
        \left[
            \int\limits_{0}^{2} \frac{1}{(x + 2 \eta)^2} x\, dx =
            - \int\limits_{0}^{2}  x\, d\frac{1}{x + 2 \eta} =
            -\frac{2}{2 + 2 \eta} + \int\limits_{0}^{2} \frac{1}{x + 2 \eta} dx =
            -\frac{1}{1 + \eta} + \ln \frac{2 +  2 \eta}{2 \eta}
        \right]
        = \\ =
        2 \pi \frac{\rho N_AZ(Z+1)}{M}
        \frac{(T+1)^2 r_e^2}{T^2(T+2)^2}
        \left(
            \ln \left( \frac{1}{\eta} + 1 \right) - \frac{1}{\eta + 1}
        \right).
    \end{gathered}
\end{equation}

$\bar{\eps}$ находится по формуле Бете:
\begin{equation}
    \bar{\eps} =
    2 \pi \frac{\rho N_AZ}{M}
    \frac{(T+1)^2 r_e^2}{T(T+2)}
    \left(
        2 \ln
        \frac{1{,}166\,T}{J/mc^2}
    \right),
\end{equation}
где
\begin{equation}
    J [\text{эВ}] =
    \begin{cases}
        13{,}6 Z & Z<10, \\
        (9{,}76 + 58{,}8 Z^{-1{,}19}) Z & Z\geqslant 10.
    \end{cases}
\end{equation}

\section{Табличный метод}

Формулы \eqref{eq: sharp} и \eqref{eq: ltr}, данные из таблиц Иоффе.
\end{document}
